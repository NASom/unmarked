\message{ !name(unmarked.Rnw.tex)}
\message{ !name(unmarked.Rnw) !offset(47) }


\abstract{Unmarked aims to be a complete environment for the
statistical analysis of wildlife data.  Currently, the focus is on
2-level hiearchical models that separately model a latent state and an
observation process.  All model parameters are estimated using with
the maximum likelihood estimates.  Unmarked uses S4 classes to help the user
organize, explore, and analyze their data in a familiar manner.}

\section{Overview of unmarked}

The goal of much wildlife research is to understand whether a
particular species is becoming more or less prevalent and if so,
determine which factors might be related to the change.  Wildlife
researchers have developed a suite of techniques to sample animal
populations.  Each of these sampling methods requires specialized
statistical model to describe the how the data are generated in terms
of interesting biological parameters such as abundance or occurrence
rates.  Thus, there is a large and growing body or statistical models
for wildlife data.  The software trend for analyzing wildlife data is
to create entirely new programs for each model or category of models.
This paradigm requires wildlife researchers to learn new software for
each new data sampling method they use.  However, many wildlife
researchers are already familiar with the popular free statistical
computing environment R \citep{R} and use it for preprocessing or
data exploration.  So it is natural to develop an R package that
analyzes a wide variety of wildlife data.  This is the goal of unmarked.

In this paper, we will first introduce wildlife sampling broadly and
how unmarked fits into this framework.  Then, we will
give a brief summary of many of the models unmarked fits.  Next, we
will describe the general syntax for using unmarked and general
principles that unmarked follows.  Finally, we will show an extended
usage example.

\subsection{Current scope of unmarked}

Wildlife sampling methods can coarsely be divided into two
broad categories: individual-based methods and site-based methods.
The most popular individual-based method is mark-recapture, which can
provide detailed information about the survival rates of individuals
and what factors are related to their survival.  Mark-capture sampling
can be expensive relative to cheaper site-based methods, which involve
visiting a sample of sites and recording some proxy to abundance.
These site-based methods provide less information than
individual-based sampling, but their reduced resource cost can allow
researchers to study species over much broader areas to understand
more global trends.  Currently, unmarked can analyze data from many
site-based sampling protocols.


\section{Models Implemented in unmarked}

Unmarked finds maximum likelihood estimates for the parameters of
many popular 2-stage hierarchical models for site sampling
protocols.  These hierarchies are defined by a latent site-level \emph{state}
model and a conditional observation-level \emph{data} model.  The state model
describes the distribution of some quantity of interest, such as the
binary variable indicating site
occupancy or positive integer describing number of animal abundance at
a site.  The data model describes the distribution of the observed data
conditional of the latent state variable.  Together, these modeling
layers for a fully parametric model for the data.  The flexibility in
the state and observation models allows generates a large class of
models that can adequately describe data collected from many animal
sampling techniques, giving unmarked its power.  This section provides
a summary of several wildlife sampling techniques and how they fit
into this 2-stage hierarchy.

\subsection{Models for Occurrence Data} \label{sec:occ}

A popular wildlife research goal is to estimate the proportion of
suitable sites that are used by the study species, called occupancy
rate or factors that affect the probability of occupancy.  To estimate
the occupancy rate, researchers can employ so-called occurrence
sampling, where surveyors visit a sample of sites and record the
binary measurement of species detection (1) or non-detection (0).
False non-detections can occur when a surveyor records a 0 for a site
that is in fact the site occupied.  Such imperfect detection can occur
because the study species are elusive or surveyor error.  Regardless
of the cause of false non-detections, failure to account for them can
lead to biased estimates of the occupancy rate.  Therefore, sampling
protocols have been developed along with accompanying statistical
models for estimating occupancy while accounting for imperfect
detection \citep{mackenzie_estimating_2002}.

Surveyors visit sites repeatedly during a season.  The occupancy state
is assumed to remain constant throughout the season.  Thus, the repeat
visits give information about the detection process.  Then, a
hierarchical Bernoulli model describes the data.  Let $\psi_i$ be the
probability occupancy at site $i$.  And $p_{ij}$ is the probability of
detection at site $i$ during the $j$th observation given that site $i$
is truly occupied.  Then,
\begin{gather}
Z_i \sim \Bern(\psi_i) \\
Y_{ij}|Z_i \sim \Bern(Z_i p_{ij}).
\end{gather}
where $Z_i$ is the partially observed occupancy state and $p_{ij}$ is
the conditional probability of detection.

Variables that are believed to affect the occupancy process are
modeled as
\begin{gather}
  \logit(\psi_i) = \bm x_i' \bm \beta,
\end{gather}
where $\bm x_i$ is a vector of site-level covariates and $\bm \beta$
is a vector of their corresponding effect parameters.  Similarly, the
probability of detection can be modeled with
\begin{gather}
  \logit(p_{ij}) = \bm v_{ij}' \bm \alpha,
\end{gather}
where $\bm v_{ij}$ is a vector of observation-level covariates and
$\bm \alpha$ is a vector of their corresponding effect parameters.



\subsection{Models for Point Count Data}

In some situations, biologists are interested in a measure of
prevalence at the site level that is finer than occupancy.  This
number is known as abundance.  One method to estimate abundance
is to visit a sample of sites and record the number of unique
individuals observed at each site, so-called point counts.  As with occurrence sampling,
imperfect detection can result in smaller observed counts than the
true number of animals at a site.  Again, to maintain identifiability
of detection parameters, repeat visits are made to each site during a
year.  Then, the following 2 level hierarchical structure describes
the data \citep{royle_n-mixture_2004}.  Let $N_i$ is the unobserved
total number of animals using a site and $Y_{ij}$ is the number of
animals observed.
\begin{gather}
  N_i \sim \Poi(\lambda_i) \label{eq:pc2} \\
  Y_{ij} \sim \Bin(N_i, p_{ij}),
\end{gather}
where $\lambda_i$ is the abundance rate and $p_{ij}$ is the detection probability.
Often $\lambda$ is of primary interest if no covariates of abundance are present.
When using covariates, interest may lay in how the variables are related to
abundance.  Other distrubions may be used in equation~\ref{eq:pc2}.  A common
alternative to the Poisson is a negative binomial distribution if overdispersion is
suspected.

\subsection{Models for General Multinomial-Poisson Data}

Finally, we consider a 2-stage model that encompasses many popular wildlife sampling methods,
the multinomial-Poisson model \citep{royle_generalized_2004}.  The general form of this model is
\begin{gather}
  N_i \sim \Poi(\lambda_i) \label{eq:mp2} \\
  \bm Y_i \sim \MN(N_i, \bm \pi_i),
\end{gather}
where $N_i$ is the latent abundance as with the point count model, and
$\bm \pi_i$ is the vector of cell probabilities governing the
observation vector at a site.  In general, $\bm \pi_i$ is determined by the
specific sampling method.  However, the final cell probability $\pi_{iJ}$ is
always the probability of an individual at site $i$ escaping detection.

\subsubsection{Distance Sampling}

In distance sampling, surveyors sample sites by traveling along a
transect or standing at a so-called point transect and recording the
distance to each detected study animal
\citep{buckland_introduction_2001}.  The continuous distances to each
observed animal are then binned into categories according to
cut-points $c_1, c_2,\dots,c_{J}$.  The user must specify a
detection pdf $g(x;\theta)$ that describes the probability of
observing an animal as a function of distance $x$ from the transect.
Then  $Y_{ij}$ is the count of observations at site $i$ falling into
the $j$th distance bin.   The cell probabilities $\bm \pi_i$ are
constructed by integrating $g$ over the distance classes.


\subsubsection{Removal Sampling}

Popular in fisheries, removal sampling is implemented by
visiting the same site $J-1$ times and trapping and removing individuals each time
with the same effort.  Thus, $Y_{ij}$ is the number
of individuals captured at the $j$th visit for $j=1,2,\dots,J-1$.

Then, the probability of an individual at site $i$ being captured on the first
visit is $\pi_{i1} = p_{i1}$.  The probability of capture on the $J$th visit is
\begin{equation}
  \pi_{iJ} = \prod_{j=1}^{J-1}(1-p_{ij})p_{iJ},
\end{equation}
and the probability of not being sampled is
\begin{equation}
  \pi_{i,J+1} = \prod_{j=1}^{J}(1-p_{ij})
\end{equation}
Or
\begin{equation}
  \bm \pi_i =
  \begin{pmatrix}
    p_{i1} \\
    (1-p_{i1})p_{i2} \\
    \vdots \\
    \prod_{j=1}^{J-1}(1-p_{ij})p_{iJ} \\
     \prod_{j=1}^{J}(1-p_{ij})
  \end{pmatrix}
\end{equation}

\subsubsection{Double observer sampling}

Double observer sampling collects data by a team of 2 surveyors
simultaneously visiting a site.  They each collect lists of identified
animals.  Thus the 4 cells of $\bm Y_i$ correspond to the numbers of
individuals seen by observer 1, observer 2, both, and neither.  Thus,
we have
\begin{equation}
  \bm \pi_i =
  \begin{pmatrix}
    p_{i1}(1-p_{i2}) \\
    (1-p_{i1})p_{i2} \\
    p_{i1}p_{i2} \\
    (1-p_{i1})(1-p_{i2})
  \end{pmatrix}
\end{equation}

\citep{royle_modeling_2004}

\subsection{Colonization-extinction model}

Wildlife scientists are often concerned with changes in state over
time.  To obtain information about change in the occupancy state
variable over time, surveyors conduct repeat occupancy studies (see
section~\ref{sec:occ}) to the same sample of sites over consecutive
seasons \citep{mackenzie_estimating_2003}.  They then seek to estimate
probabilities of colonization ($\gamma_{it}$) and extinction
($\epsilon_{it}$), where colonization is the change of an unoccupied
site to occupied and extinction if the change of an occupied site to
unoccupied.  If the occupancy status is assumed to evolve according to
a Markov process, then a 2-state finite hidden Markov model describes
these data.  Let $Y_{itj}$ denote the observed animal occurrence
status at visit $j$ during season $t$ to site $i$.  Then
\begin{gather}
  Z_{i1} \sim \Bern(\psi) \\
  Z_{it} \sim
  \begin{cases}
    \Bern(\gamma_{i(t-1)}) & \text{if $Z_{i(t-1)} = 0$} \\
    \Bern(1-\epsilon_{i(t-1)}) & \text{if $Z_{i(t-1)} = 1$}
  \end{cases}, t=2,3,\dots,T \\
  Y_{itj} | Z_{it} \sim \Bern(Z_{it} p_{itj})
\end{gather}


\section{The unmarked implementation}

Unmarked can fit all of the above models.  By exploiting the common
2-stage hierarchical structure of these models, unmarked
implements common data structures, fitting syntax, and post-fit
processing to form a cohesive framework for site-based wildlife data
analysis.  The modern S4 class system
\citep{chambers_software_2008} provides a convenient mechanism to implement this
common structure.

\begin{table}
  \caption{Models fit by the stable version of unmarked.}
\begin{tabular}{cc}

\end{tabular}
\end{table}

\subsection{Data Requirements}

Because of the multi-level structure of the models, standard data
structures such as data frames or matrices are inadequate for these
animal sampling methods.  Also, good programming practice holds all
data used in an analysis in one portable object to simplify future
reference to previous analyses \citep{gentleman_r_2008}.  Repeated fitting
calls using the same set of data require less code repetition if all
data is contained in a single object.  Finally, calls to fitting
functions have a cleaner appearance with a more obvious purpose when
the call is not overfull with data arguments.  For these reasons, we
developed S4 data structures to contain data for use with unmarked's
model-fitting functions.  Although this necessitates the extra step of
creating the data object before fitting models, the benefits outweigh
this minor inconvenience.  The parent data class is called an
unmarkedFrame and all other unmarked data types inherit from it.

An unmarkedFrame object contains slots for an observation matrix
*y*, a data frame of site-level covariates *siteCovs*, and a data
frame of observation-level covariates *obsCovs*.  The matrix of
observed response variables, *y*, is the only required unmarkedFrame
slot.  *y* is an $M \times J$ matrix for a data set with $M$ sites
and $J$-dimensional $\bm y_i$.  Thus, for occurrence data, $R$

The meaning of observation vectors varies by data type.  For
example, in occurrence data, $R$ is the number of repeat site visits
and elements of the observation vectors are the repeated sampling
events at each site.  In double observer sampling, observation vectors
correspond to the multinomial response categories for counts for
observer 1, observer 2, and both observers.  For distance sampling,
$y_i$ is the vector of counts falling into the $R$ distance classes
specified by the user.

unmarkedFrames also contain covariate data, measured either at the
observation level or at the site level.  The slots obsCovs and
siteCovs hold these data respectively.

Automatic handling of missing data via obsToY matrix...

Other data components are unique to specific types of unmarkedFrames.

Unmarked aims to make importing data into unmarkedFrames easy, through
the use of several import functions.

<<>>=
library(unmarked)
data(mallard)
head(mallard.y)
str(mallard.site)
str(mallard.obs)
mallardUMF <- unmarkedFramePCount(y = mallard.y, siteCovs = mallard.site, obsCovs = mallard.obs)
head(mallardUMF)
@

unmarked can conveniently return an unmarkedFrame in data frame format
by simply showing the object.

%<<echo=FALSE>>=
%mallardUMF
%@


\subsection{Pre-Analysis Data Exploration}

Plotting data is a useful first step when embarking on any statistical analysis.
Unmarked implements plot methods for each of the unmarkedFrame data types.

<<include=FALSE>>=
plot(mallardUMF)
@


Unmarked also has summary methods tailored for unmarkedFrames.

<<>>=
summary(mallardUMF)
@


\subsection{Fitting models}

Each model type has a corresponding fit function.  For example, to fit a point count model, we call pcount.

<<eval=FALSE>>=
fm <- pcount(~ ivel+ date + I(date^2) ~ length + elev + forest, mallardUMF)
@


\subsection{Examining model fits}

\subsection{Model Selection}


\subsubsection{Goodness-of-fit}


\subsubsection{Prediction}



\section{Extensive Examples}


\subsection{Occurrence Data}


\subsection{Point Count Data}


\subsection{Distance Data}

\section{Future directions for unmarked development}

Spatial models, Bayesian implementations,...





\bibliographystyle{apa}
\bibliography{unmarked}


\end{document}

\message{ !name(unmarked.Rnw.tex) !offset(-387) }
